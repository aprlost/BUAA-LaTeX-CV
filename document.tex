\documentclass[11pt, a4paper]{article}

%----------------------------------------------------------------------------------------
%	宏包导入
%----------------------------------------------------------------------------------------
% [页面边距设置]
% top/bottom: 上下边距。如果内容写不下,可以将 1.5cm 改为 1.0cm
% left/right: 左右边距。
\usepackage[top=1.5cm, bottom=1.5cm, left=2cm, right=2cm]{geometry}

\usepackage{xeCJK}       % 中文支持(必须使用 xelatex 编译)
\usepackage{graphicx}    % 图片插入
\usepackage{enumitem}    % 列表环境控制 (调整 itemize 的间距)
\usepackage{titlesec}    % 标题格式控制 (调整 section 的字体和间距)
\usepackage{xcolor}      % 颜色控制
\usepackage{hyperref}    % 超链接
\usepackage{tikz}        % 绘图 (用于 Logo 的绝对定位)
\usepackage{tasks}       % 任务列表
\usepackage{multicol}    % 多栏排版
\usepackage{calc}        % 长度计算
\usepackage{tabularx}    % 增强表格

%----------------------------------------------------------------------------------------
%	字体与格式设置
%----------------------------------------------------------------------------------------
\setmainfont{Times New Roman} % 西文主字体 (数字和英文)

\hypersetup{
	hidelinks, % 隐藏链接的边框和颜色(打印时不显示蓝色框)
}

% [中文设置]
% Windows 用户保持默认即可 (SimSun=宋体, SimHei=黑体)
% Mac 用户如果编译报错,请注释掉下面一行,解开 Songti SC 那行的注释
\setCJKmainfont[BoldFont=SimHei, ItalicFont=KaiTi]{SimSun}
% \setCJKmainfont[BoldFont=Songti SC Bold, ItalicFont=Kaiti SC]{Songti SC}

% 定义“楷体”命令
% AutoFakeBold=2.5 表示如果找不到粗楷体,就由机器自动加粗 2.5 倍
\newCJKfontfamily\kaishu[AutoFakeBold=2.5]{KaiTi}

\pagestyle{empty} % 去掉页码 (单页简历不需要页码)

% [Section 标题格式定制]
% \titlespacing*{命令}{左缩进}{段前距}{段后距}
% 想要标题和正文挤得更紧一点?把 {6pt}{4pt} 改为 {3pt}{2pt}
\titleformat{\section}
{\Large\kaishu\bfseries\raggedright} % 格式:大号、楷体、加粗、左对齐
{}{0em}
{}
[\titlerule] % 标题下方的横线
\titlespacing*{\section}{0pt}{6pt}{4pt}

% [列表环境全局压缩]
% parsep: 段落间距; itemsep: 条目间距; topsep: 列表与上方文本间距
% 这里的设置影响全局的 itemize。如果觉得太挤,把 1pt 改大。
\setlist[itemize]{parsep=0pt, itemsep=1pt, topsep=1pt, leftmargin=1.5em}

%----------------------------------------------------------------------------------------
%	Logo设置 (使用 TikZ 绝对定位)
%----------------------------------------------------------------------------------------
\newcommand{\placelogo}{%
	\begin{tikzpicture}[remember picture,overlay]
		% [Logo 位置微调]
		% xshift: 水平偏移 (正数右移,负数左移)
		% yshift: 垂直偏移 (正数上移,负数下移)
		% 这里的 -0.4cm 决定了 Logo 离顶部的距离。想上移就改成 0cm 或正数。
		\node[anchor=north west, inner sep=0pt, xshift=1.5cm, yshift=-0.4cm] at (current page.north west) {
			% [Logo 大小] height 控制高度,keepaspectratio 保持长宽比
			% 如果没有图片文件,请暂时注释掉下面这行
			\includegraphics[height=1.5cm, keepaspectratio]{logo.png}
		};
	\end{tikzpicture}%
}

%----------------------------------------------------------------------------------------
%	自定义项目行宏
%----------------------------------------------------------------------------------------
% 用于 "项目名称 - 角色 - 时间" 的三栏排版
\newcommand{\projectitem}[3]{%
	\noindent%
	\begin{minipage}[t]{0.60\linewidth}
		\textbf{#1}
	\end{minipage}%
	\begin{minipage}[t]{0.20\linewidth}
		\centering \textbf{#2}
	\end{minipage}%
	\begin{minipage}[t]{0.20\linewidth}
		\raggedleft #3
	\end{minipage}%
	\par\vspace{0.1em}% 行后的微小间距
}

%----------------------------------------------------------------------------------------
%	正文开始
%----------------------------------------------------------------------------------------
\begin{document}
	
	\placelogo % 放置左上角的 Logo
	
	%--- 头部信息 ---
	\noindent
	% [左侧文字区域] 宽度 0.73\textwidth
	\begin{minipage}[b]{0.73\textwidth}
		{\huge \kaishu \textbf{奶龙}} % 姓名
		
		\vspace{0.4em} % 姓名与下方信息的距离
		
		% 个人信息表格
		\begin{tabular}{@{}ll}
			\textbf{电话:}138-0000-0000 & \textbf{邮箱:}zhangsan@example.edu.cn \\
			\textbf{出生年月:}2002.08 & \textbf{籍贯:}江苏南京 \\
			\textbf{政治面貌:}预备党员 & \textbf{英语水平:}CET-6 (610分) \\
		\end{tabular}
	\end{minipage}%
	\hfill % 填充空白,把左右两栏撑开
	% [右侧照片区域] 宽度 0.24\textwidth
	\begin{minipage}[b]{0.24\textwidth}
		\raggedleft % 图片右对齐
		\raisebox{0pt}[0pt][0pt]{%
			% [照片大小] width 控制宽度
			% 如果没有 photo.jpg,请先注释掉下面这行以免报错
			\includegraphics[width=3.2cm, keepaspectratio]{photo.jpg}%
		}
	\end{minipage}
	
	% [头部与正文的间距]
	% 负值表示向上拉,减小空白。如果觉得头部太挤,可以改为 -0.5em 或 0em
	\vspace{-0.75em}
	
	%============================================================================
	\section{教育经历}
	
	\noindent
	% [教育经历表格]
	% l @{\hskip 1.5em} 表示左对齐列,并在右侧强制插入 1.5em 的空白
	% @{\extracolsep{\fill}} 让最后一列(时间)自动推到最右边
	\begin{tabular*}{\textwidth}{l @{\hskip 1.5em} l @{\hskip 1.5em} l l @{\extracolsep{\fill}} r@{}}
		\textbf{北京航空航天大学} & 奶龙学院 & \makebox[9em][l]{\textbf{虚拟奶龙技术}} & 本科 & 2023.09 -- 至今 \\
	\end{tabular*}
	
	\begin{itemize}
		\item \textbf{核心课程:}机器学习导论(98) \enspace 深度学习(96) \enspace 计算机视觉(95) \enspace 自然语言处理(97) \enspace 算法设计与分析(94) \enspace 概率论与数理统计(99)
		\item \textbf{学业成绩:}GPA:3.9/4.0 \quad \textbf{专业排名:}3/120(前2.5\%)
	\end{itemize}
	
	%============================================================================
	\section{科研经历}
	
	\projectitem{基于多传感器融合的自动驾驶感知系统}{项目负责人}{2024.03 -- 2024.09}
	\begin{itemize}
		\item \textbf{项目描述:} 针对自动驾驶在恶劣天气下感知能力下降的问题,设计了一套基于 \textbf{LiDAR} 与 \textbf{Camera} 融合的感知系统。
		\item \textbf{工作内容:}
		\begin{itemize}
			\item 搭建了基于 \textbf{ROS2} 的数据采集平台,完成了 KITTI 格式数据集的标注工作。
			\item 提出了 \textbf{MV-Fusion} 算法,在特征层对点云与图像进行融合,有效提升了小目标检测精度。
		\end{itemize}
	\end{itemize}
	
	\projectitem{面向医疗影像的少样本分割网络(Med-Seg)\hspace{1em}\raisebox{-0.2em}{\href{www.baidu.com}{\includegraphics[height=1.5em]{eye.png}}}}{第一作者}{2023.09 -- 2023.12}
	\begin{itemize}
		\item \textbf{项目背景:} 医疗数据标注成本高昂,针对小样本场景下的分割失效问题,提出了一种基于原型网络的分割框架。
		\item \textbf{主要贡献:} 引入了\textbf{注意力机制}对原型特征进行校准,减少了噪声干扰;使用 \textbf{Python/PyTorch} 复现了 DeepLabv3+ 作为基准对比,结果显示 Dice 系数提升 \textbf{3.2\%}。
		\item \textbf{成果:} 论文已投稿至 MICCAI 2024 (在审)。
	\end{itemize}
	
	%============================================================================
	\section{项目经历/荣誉奖项}
	
	\noindent \textbf{中国大学生计算机设计大赛} \hfill \textbf{国家级二等奖} \quad 2026.05
	% [特定列表的间距调整]
	% nosep: 极度压缩,取消所有垂直间距
	% topsep: 列表与上方文字的微调
	\begin{itemize}[nosep, leftmargin=2em, topsep=2pt]
		\item \textbf{项目内容:}开发了一款基于 \textbf{Flutter} 的视障人士辅助出行 APP,集成了物体识别与语音导航功能。
	\end{itemize}
	
	% \vspace{0.2em} % 如果觉得奖项之间太拥挤,可以手动取消注释加一点间距
	\noindent \textbf{美国大学生数学建模竞赛 (MCM/ICM)} \hfill \textbf{Finalist (特等奖提名)} \quad 2026.01
	
	\noindent \textbf{国家奖学金} \hfill \textbf{前 0.2\%} \quad 2025.12
	
	\noindent \textbf{学科竞赛奖学金} \hfill \textbf{一等奖} \quad 2025.12
	
	\noindent \textbf{学习优秀奖学金} \hfill \textbf{一等奖} \quad 2025.12
	
	\noindent \textbf{第十七届全国大学生数学竞赛(非数学A类)} \hfill \textbf{省级一等奖} \quad 2025.12
	
	\noindent \textbf{学习优秀奖学金} \hfill \textbf{一等奖} \quad 2024.12
	
	\noindent \textbf{第十六届全国大学生数学竞赛(非数学A类)} \hfill \textbf{省级一等奖} \quad 2024.12
	
	%============================================================================
	\section{综合发展}
	\begin{itemize}
		\item \textbf{开发工具:}熟练使用 Git, Docker, Linux (Ubuntu), LaTeX。
		\item \textbf{其他:}志愿时长学生工作等。
	\end{itemize}
	
\end{document}